\documentclass[a5paper]{article}
\usepackage{geometry}
\geometry{landscape, margin=12mm}
\usepackage{amsmath}
\usepackage{listings}
\usepackage[usenames.dvipsnames]{xcolor}
\definecolor{k}{HTML}{F94144}
\definecolor{o}{HTML}{E8960E}
\definecolor{zh}{HTML}{F9C74F}
\definecolor{z}{HTML}{90BE6D}
\definecolor{g}{HTML}{4D908E}
\definecolor{s}{HTML}{277DA1}
\definecolor{f}{HTML}{AA3DAA}
\definecolor{gray}{HTML}{888888}
\definecolor{Blue}{HTML}{8888FF}
%\definecolor{gray}{HTML}{F0F0F6}
%\definecolor{gray}{HTML}{E4E8EE}

\newcommand{\D}[2]{\frac{\partial #1}{\partial #2}}

\lstset{language=C, basicstyle=\ttfamily, basewidth=0.46em,commentstyle=\color{gray},stringstyle=\color{Red},keywordstyle=\color{Blue},breaklines=true,morekeywords={cvar,calc}, showstringspaces=false} 
\begin{document}
%\pagecolor{gray}

\section{Polynomials}

$P_M$ is the order $M$ (shifted) Legendre polynomial on the range $[0,1]$.

It has $M$ roots: $0<\xi_i<1$, $i=0..M-1$.

Gauss Legendre quadrature can be used:

\begin{equation} \label{eq:gaussq}
  \int_0^1 d\xi f(\xi) = \sum_{i=0}^{M-1} w_i f(\xi_i) 
\end{equation}

It is accurate if $f(\xi)$ is a polynomial of order less or equal to $2M-1$. 

Let us construct $M$ Lagrange polynomials $\psi_i$, $i=0..M-1$, interpolating the roots:
\begin{equation}
  \psi_j(\xi_i) = \delta_{ij}.
\end{equation}
Found as
\begin{equation}
  \psi_j(\xi) = \prod_{i\ne j} \frac{x-x_i}{x_j-x_i}.
\end{equation}
Their order is $M-1$.

They can be integrated:
\begin{equation}
  \int_0^1 d\xi \psi_l(\xi) = \sum_{i=0}^{M-1} w_i \psi_l(\xi_i)  = w_l
\end{equation}
and this is how the Gauss weights are found. 

Orthogonality works as
\begin{equation}
  \int_0^1 d\xi \psi_l(\xi) \psi_k(\xi) = \sum_{i=0}^{M-1} w_i \psi_l(\xi_i) \psi_k(\xi_i) = w_l \delta_{kl}.
\end{equation}

And one more useful expression:
\begin{equation}
  \int_0^1 d\xi \D{\psi_l(\xi)}\xi \psi_k(\xi) = \sum_{i=0}^{M-1} w_i \left. \D{\psi_l}{\xi}\right|_{\xi_i} \psi_k(\xi_i) = w_k \left. \D{\psi_l}{\xi}\right|_{\xi_k}.
\end{equation}

To project a function $f(\xi)$ onto the basis
\begin{equation}
  f(\xi) = \sum_{l=0}^{M-1} f_l \psi(\xi) 
\end{equation}
the coefficients are found as
\begin{equation}
  \label{eq:takerootvalue}
  f_l = f(\xi_l).
\end{equation}

\clearpage
\section{DG} \label{sec:DG}
Let us take a system of equations  for vector of variables $\color{o}{\mathbf{u}}$, its components are $u_p$, $p=1..N_Q$. 
\begin{equation} \label{init}
 \D {\color{o} u_p} {t} + 
 \D {\color{o} F_p (\pmb u)}{x} = 
  {\color{o} S_p(\pmb u, x, t)} \quad
\end{equation}

Let us take an element of spacetime $t^{(k)}<t<t^{(k)}+\Delta t$, $x^{(m)}<x<x^{(m)}+\Delta x$ and introduce local dimensionless variables in it
\begin{align}
  \tau^{(k)}(t) = \frac{t-t^{(k)}}{\Delta t}; \quad t(\tau^{(k)}) = t^{(k)}+\tau \Delta t \\
  \xi^{(m)}(x) = \frac{x-x^{(m)}}{\Delta x}; \quad x(\xi^{(m)}) = x^{(m)}+\xi {\Delta x}. \\
  \xi^{(m)} \equiv \xi; \quad  \tau^{(k)} \equiv \tau.
\end{align}
The last expressions show how we omit the indexes of the current element. The indexes are written only for other elements, such as $(m+1)$ for its neighbor. 

We store the solution in each element as the decomposition on the basis of Lagrange polynomials
$
  u_p \approx  {\color{f} \mathsf{w}_{pl_\xi} }
   {\color{o} \psi_{l_\xi} (\xi(x))}
  $, $l_\xi = 1... N_{Bx}$, $N_{Bx} = M_x +1$ where $M_x$ is the order of the spatial polynomial basis. 
The coefficients ${\color{f} \mathsf{w}_{pl_{\xi}}}$ are initialized with (\ref{eq:takerootvalue}) and stored on the mesh.
\begin{equation} 
  {\color{f} \mathsf{w}_{pl_{\xi}}} =  {\color{o} u_p(x=x(\xi_{l_{\xi}}), t=0)}.
\end{equation}

We introduce dimensionless fluxes and sources as:
\begin{equation} 
  \pmb F^* = \frac{\Delta t}{\Delta x} \pmb F; \quad
  \pmb S^* =       \Delta t            \pmb S.
\end{equation}

The system with   $\pmb F^*$ and $\pmb S^*$ is as follows. We multiply it by a basis function (Lagrange polynomial) and integrate over the element. 
\begin{equation} \label{DG}
\left.
\left.
 \D {\color{o} u_p} {\tau} + 
 \D {\color{o} F^*_p (\pmb u)}{\xi} = 
  {\color{o} S^*_p(\pmb u, \xi(x), \tau(t))} \quad
 \right| 
 {\color{z} \cdot
  \psi_{k_\xi} (\xi) }
 \right| 
 {\color{z} \cdot 
  \int_0^1 d\xi
  \int_0^1 d\tau}
\end{equation}

Below we omit the asterisk. Just remember that the sources and fluxes need to be made dimensionless. 


The time derivative is integrated into the difference of $\color{o} {u_p}$ at $\tau=0$ and $\color{o} {u_p}$ at $1$.

\begin{equation}
  \int_0^1 \D {u_p}{\tau} d\tau = \left. u_p \right|_{\tau=1} -  \left. u_p \right|_{\tau=0}.
\end{equation}

The solution $\color{o} {u_p}$ at $\tau = 0$ is known.  
We seek to build an expression for  $\color{o} {u_p}$ at $\tau = 1$.

\begin{align} 
  \left. u_p \right|_{\tau=0} \approx 
  {\color{f} \mathsf{w}_{pl_\xi} } 
  {\color{o} \psi_{l_\xi} (\xi(x)) }\\
  \left. u_p \right|_{\tau=1} \approx 
  {\color{k} \mathsf{u}^+_{pl_\xi} } 
  {\color{o} \psi_{l_\xi} (\xi(x)) }
\end{align}

Below is the usual expression for finite element methods, which is obtained by integrating the flux term by parts. 
We introduce  ${\color{s} \mathsf{F}_\text{right}}$ and ${\color{s} \mathsf{F}_\text{left}}$ that express the flux over the element boundary. 
Instead of evaluating these as  $F_p (\pmb u)$, the expression for a solution of the Riemann problem is taken.
\begin{align}
\begin{split}
  {\color{k} \mathsf{u}^+_{pl_\xi} }
  \langle {\color{o} \psi_{l_\xi} (\xi(x)) } 
  {\color{z} \psi_{k_\xi} (\xi) } \rangle -
  &  {\color{f} \mathsf{w}_{pl_\xi} } 
  \langle {\color{o} \psi_{l_\xi} (\xi(x)) } 
  {\color{z} \psi_{k_\xi} (\xi) } \rangle + 
  {\color{z} \int_0^1 d\tau \left( 
  \left.
  {\color{s} \mathsf{F}_\text{right}  }
  \psi_{k_\xi} 
  \right|_{\xi=1}
  - 
  \left.
  {\color{s} \mathsf{F}_\text{left}  }
  \psi_{k_\xi} 
  \right|_{\xi=0}
  -
  \int_0^1 
  {\color{o} F_p (\pmb u)}
  \D{\psi_{k_\xi}}{\xi}
  d\xi
  \right) = } \\ =
  &{\color{z}  
  \int_0^1 d\xi
  \psi_{k_\xi} (\xi) 
  \int_0^1 d\tau}
  {\color{o} S_p}.
\end{split}
\end{align}

We get the explicit expression for the sought value. 
Now we need to decide on the expression for fluxes and evaluate the integrals on the right side.
\begin{equation}
  {\color{k} \mathsf{u}^+_{pk_\xi} }
  w_{k_\xi}  = 
  {\color{f} \mathsf{w}_{pk_\xi} } 
  w_{k_\xi}  - 
  {\color{z} \int_0^1 d\tau \left( 
  \left.
  {\color{s} \mathsf{F}_\text{right}  }
  \psi_{k_\xi} 
  \right|_{\xi=1}
  - 
  \left.
  {\color{s} \mathsf{F}_\text{left}  }
  \psi_{k_\xi} 
  \right|_{\xi=0}
  -
  \int_0^1 
  {\color{o} F_p (\pmb u)}
  \D{\psi_{k_\xi}}{\xi}
  d\xi
  \right) + }
 {\color{z}  
  \int_0^1 d\xi
  \psi_{k_\xi} (\xi) 
  \int_0^1 d\tau}
  {\color{o} S_p}.
\end{equation}
\section{ADER} \label{sec:ADER}

The integrals in time can be computed after we use a predictor for $u$ in the $t^{(k)}<t<t^{(k)}+\Delta t$ range. 
Let us integrate system again, similarly to (\ref{DG}), but with ${\color{z} \cdot  \theta_k (\xi,\tau) }$ such that :
\begin{equation} \label{localADERinit}
\left.
\left.
 \D {\color{o} u_p} {\tau} + 
 \D {\color{o} F^*_p (\pmb u)}{\xi} = 
 {\color{o} S^*_p} \quad
 \right| 
 {\color{z} \cdot
  \theta_k (\xi,\tau) }
 \right| 
 {\color{z} \cdot \int_0^1 d\xi
  \int_0^1 d\tau }
\end{equation}

Here, $l_\tau = 1... N_{Bt}$, $N_{Bt} = M_t +1$ where $M_x$ is the order of the temporal polynomial basis. 

The expansion coefficients for sources and fluxes are evaluated as functions of $\pmb u$. 
This is correct when the conditions for the Gauss quadrature (\ref{eq:gaussq}) are satisfied. 
In other cases, we assume the expression is sufficiently precise. 
\begin{equation}
 {\color{o} u_p} \approx 
 {\color{k} q_p (\xi,\tau)} = 
 {\color{k} q_{pl_\xi l_\tau} } 
 {\color{o} \psi_{l_\xi}(\xi)
            \psi_{l_\tau}(\tau) } = 
 {\color{k} q_{pl_\xi l_\tau} } 
 {\color{o} \theta_{l}(\xi,\tau) }
\end{equation}
\begin{equation}
{\color{z} \theta_l (\xi,\tau)= \psi_{l_\xi}(\xi)
            \psi_{l_\tau}(\tau) }
\end{equation}

\begin{equation}
{\color{o} F^*_p(\pmb u)} = 
{\color{k} \mathcal{F}_{pl} }
{\color{o} \theta_{l}(\xi,\tau) }; \quad {\color{k} \mathcal{F}_{pl}  \approx  F^*_p(\pmb q) }
\end{equation}

\begin{equation}
{\color{o} S^*_p(\pmb u)} = 
{\color{k} \mathcal{S}_{pl} }
{\color{o} \theta_{l}(\xi,\tau) }; \quad {\color{k} \mathcal{S}_{pl}  \approx  S_p(\pmb q) }
\end{equation}

We need to find ${\color{k} q_{pl_\xi l_\tau} }$. 
The initial condition at $\tau=0$ has to be taken into account. 
Let us integrate the term with the time derivative by parts:

\begin{equation}
 {\color{z} \int_0^1 \int_0^1 }
 \D {\color{o} u_p} {\tau} 
 {\color{z} \theta_k (\xi,\tau) }
 {\color{z} d\xi d\tau } =
 {\color{z}
   \int_0^1 \left( 
     {\color{o} u_p} 
     \left. \theta_{k} \right|_{\tau=0}^{\tau=1} - 
     \int_0^1 {\color{o} u_p}
     \D{\theta_k}{\tau} d\tau
   \right) d\xi.
 }
\end{equation}

This way, we can expose the known values:

\begin{equation}
 {\color{o}
  u_p(\tau=0) } = 
  {\color{f} \mathsf{w}_{pl_\xi}}
 {\color{o} \psi_{l_\xi}(\xi) } ; \quad
  {\color{k} q_{pl}}{\color{o}\psi_{l_\tau}(0)} = {\color{f} \mathsf{w}_{pl_\xi}}
\end{equation}

\begin{equation}
 {\color{o}
  u_p(\tau=1) } = 
 {\color{k} q_{pl}}
 {\color{o} \psi_{l_\tau}(1) }
 {\color{o} \psi_{l_\xi}(\xi) }
\end{equation}

\begin{align} \begin{split}
 {\color{z} \cdot \int_0^1 \int_0^1 }
 \D {\color{o} u_p} {\tau} 
 {\color{z} \theta_k (\xi,\tau) }
  {\color{z} d\xi d\tau }  & =
 {\color{k} q_{pl}}
 {\color{o} \psi_{l_\tau}(1) }
 {\color{z} \psi_{k_\tau}(1) } 
 {\color{z} \int_0^1 }
 {\color{o} \psi_{l_\xi}(\xi) }
 {\color{z} \psi_{k_\xi}(\xi) d\xi} - 
  {\color{f} \mathsf{w}_{pl_\xi}}
 {\color{z} \psi_{k_\tau}(0) } 
 {\color{z} \int_0^1 }
 {\color{o} \psi_{l_\xi}(\xi) }
 {\color{z} \psi_{k_\xi}(\xi) d\xi} - \\ -
  &{\color{k} q_{pl}}
 {\color{z} \int_0^1 } 
 {\color{o} \psi_{l_\xi}(\xi) }
 {\color{z} \psi_{k_\xi}(\xi) d\xi} 
 {\color{z} \int_0^1 }
 {\color{o} \psi_{l_\tau}(\tau) }
 {\color{z} \D{\psi_{k_\tau}(\tau)}{\tau} d\tau} .
\end{split}\end{align}

\begin{equation}
 { \int_0^1 }
 { \psi_{l_\xi}(\xi) }
 { \psi_{k_\xi}(\xi) d\xi  = \delta_{l_\xi k_\xi} w_{k_\xi}}.
\end{equation}


Finally, from (\ref{localADERinit}), we get the system of $N_{Btx} = N_{Bx} \cdot N_{Bt}$  equations for ${\color{k} q_{pk_\xi k_\tau} }$. 
Note that the summation index is $l :\{l_\xi,l_\tau\}$, and $k$ (i.e.,  $k = N_{Bt}k_\xi+k_\tau$) is the number of the equation in the system.  

\begin{align} \begin{split}
  {\color{k} q_{pl}}
  {\color{z} \delta_{l_\xi k_\xi} w_{k_\xi}} 
  &\left(
     {\color{o} \psi_{l_\tau}(1) }
     {\color{z} \psi_{k_\tau}(1) } - 
     {\color{z} \int_0^1 }
     {\color{o} \psi_{l_\tau}(\tau) }
     {\color{z} \D{\psi_{k_\tau}(\tau)}{\tau} d\tau}
  \right) = \\ =
  &{\color{f} \mathsf{w}_{pl_\xi}}
  {\color{z} \psi_{k_\tau}(0) }
  {\color{z} \delta_{l_\xi k_\xi} w_{k_\xi}} - 
  {\color{o} \mathcal{F}_{pl_\xi l_\tau}  ( {\color{k} \pmb q_{l}})}
  {\color{z} \int_0^1 } 
  {\color{o} \psi_{l_\tau}(\tau) }
  {\color{z} \psi_{k_\tau}(\tau) d\tau} 
  {\color{z} \int_0^1 }
  {\color{o} \psi_{k_\xi}(\xi) }
  {\color{z} \D{\psi_{l_\xi}(\xi)}{\xi} d\xi} + \\ +
  &{\color{z} \int_0^1 } 
  {\color{o} \psi_{l_\tau}(\tau) }
  {\color{z} \psi_{k_\tau}(\tau) d\tau} 
  {\color{z} \int_0^1 }
  {\color{o} \psi_{k_\xi}(\xi) }
  {\color{z} {\psi_{l_\xi}(\xi)}
  {\color{o} \mathcal{S}_{pl_\xi l_\tau} ( {\color{k} \pmb q_{l}}, \xi, \tau) }
  d\xi}.
\end{split}\end{align}

\begin{align} \begin{split}
     { \int_0^1 }
     { \psi_{l_\tau}(\tau) }
     { \D{\psi_{k_\tau}(\tau)}{\tau} d\tau } = 
     \left. {w}_{l_\tau} \D{\psi_{k_\tau}}{\tau} \right|_{\tau=\tau_{l_\tau}}
\end{split}\end{align}

\begin{align} \begin{split}
  {\color{k} q_{pl}}
  { \delta_{l_\xi k_\xi} {w}_{k_\xi}} 
  \left(
     {\color{o} \psi_{l_\tau}(1) }
     {\color{z} \psi_{k_\tau}(1) } - 
     {\color{z}
     \left. {w}_{l_\tau} \D{\psi_{k_\tau}}{\tau} \right|_{\tau=\tau_{l_\tau}}
     }
  \right) = 
  {\color{f} \mathsf{w}_{pk_\xi}}
  {\color{z} \psi_{k_\tau}(0) }
  {\color{z} {w}_{k_\xi}} - 
  {\color{o} \mathcal{F}_{pl}  ( {\color{k} \pmb q_{l}})}
  \delta_{l_\tau k_\tau} {w}_{k_\tau} 
     \left. {w}_{k_\xi} \D{\psi_{l_\xi}}{\xi} \right|_{\xi=\xi_{k_\xi}} + 
  {\color{o} \mathcal{S}_{pl}  ( {\color{k} \pmb q_{l}})}
  \delta_{l_\tau k_\tau} {w}_{k_\tau} 
  \delta_{l_\xi k_\xi} {w}_{k_\xi} 
\end{split}\end{align}

We introduce new notations to shorten the expression. The $ K_{1,kl}$, $K_{\xi, kl}$ are $N_{Btx}\times N_{Ntx}$ matrices that can be found as soon as the polynomial basis is fixed. 
${\color{f} W_k }$, 
${\color{o} \mathcal{F}_{pl}  ( {\color{k} \pmb q_{l}})}$, 
and  ${\color{o} \mathcal{S}_{pk}  ( {\color{k} \pmb q_{k}})}
{w}_{k_\xi} {w}_{k_\tau} $
are vectors of $N_{Btx}$ components.  
\begin{equation}
 K_{1,kl}
 {\color{k} q_l } =
 {\color{f} W_k } - K_{\xi, kl} 
 {\color{o} \mathcal{F}_{pl}  ( {\color{k} \pmb q_{l}})} +
 {\color{o} \mathcal{S}_{pk}  ( {\color{k} \pmb q_{k}})}
{w}_{k_\xi} {w}_{k_\tau} 
\end{equation}


The system is solved by iterations. The index  $i$ is the number of iteration:
\begin{equation}\label{eq:ADERiter}
 {\color{k} q_m ^{(i+1)} } =
 (K_{1}^{-1})_{mk} \left(
 {\color{f} W_k } - K_{\xi, kl} 
 {\color{o} \mathcal{F}_{pl}  ( {\color{k} \pmb q_{l}^{(i)} })} +
 {\color{o} \mathcal{S}_{pk}  ( {\color{k} \pmb q_{k}^{(i)} })}
{w}_{k_\xi} 
{w}_{k_\tau} 
 \right)
\end{equation}

\begin{equation}
\boxed{
 {\color{k} q_m ^{(i+1)} } =
 (K_{1}^{-1})_{mk} 
 {\color{f} W_k } -
 (K_{1}^{-1} 
 K_{\xi} )_{ml}
 {\color{o} \mathcal{F}_{pl}  ( {\color{k} \pmb q_{l}^{(i)} })} +
 (K_{1}^{-1})_{mk} 
 {\color{o} \mathcal{S}_{pk}  ( {\color{k} \pmb q_{k}^{(i)} })}
{w}_{k_\xi} 
{w}_{k_\tau} 
}
\end{equation}

\section{Fluxes} \label{sec:Fluxes}
Let us introduce the Jacobian matrix:
\begin{equation}\label{eq:Adef}
  \pmb A = \D {\pmb F}{\pmb u}.
\end{equation}

Its eigendecomposition is 

\begin{equation}
  \pmb A = \pmb K(\pmb u) \pmb \Lambda (\pmb u) \pmb K^{-1}(\pmb u)  
\end{equation}
where $\pmb K$  is the matrix formed by the right eigenvectors, and $\Lambda$ is the diagonal matrix whose diagonal entries are the eigenvalues $\lambda_p$.
If the system is linear, $\pmb A $ does not depend on $\pmb u$. 
 
Let us introduce $\pmb w =  \pmb K^{-1} \pmb u $. 
The system of equations (\ref{init}) can be multiplied by $\pmb K^{-1}$ on the left and written as 
\begin{equation}
  \D  {\pmb w} t + 
\pmb \Lambda \D {\pmb w}{x} = 
  \pmb K^{-1} \pmb S
\end{equation}
If matrices and sources do not depend on $\pmb u$, these are $Q$ independent equations for $\pmb w$. 
For the $p$-th equation, its $\lambda_p$ determine the direction of the flow. 
Let us decompose $\pmb A$ and $\pmb \Lambda$ according to the flow direction:

\begin{equation}
  \pmb \Lambda = \pmb \Lambda^+ + \pmb \Lambda^- 
\end{equation}
where $\Lambda^+$ has only the positive eigenvalues $\lambda^+_p = \max(\lambda_p,0)$ on the diagonal,  and $\Lambda^+$ has only the negative eigenvalues $\lambda^-_p = \min(\lambda_p,0)$.

\begin{equation}
  \pmb A = \pmb A^+ + \pmb A^- = \pmb K \pmb \Lambda^+ \pmb K^{-1} +  \pmb K \pmb \Lambda^- \pmb K^{-1}
\end{equation}

We define $\pmb F^+$ and $\pmb F^-$ as 

\begin{equation}
  \pmb A^+ = \D {\pmb F^+}{\pmb u}, \quad  
  \pmb A^- = \D {\pmb F^-}{\pmb u}.
\end{equation}

If the system is linear, 
\begin{equation}
  \pmb F^+ = \pmb A^+ \pmb u, \quad
  \pmb F^- = \pmb A^- \pmb u.
\end{equation}

The flux to use on the boundary between element $(m)$ and $(m+1)$ is taken as 
\begin{equation}
  {\color{s} \mathsf{F}_\text{right}} = \pmb A^+ \pmb u^{(m)} + \pmb A^- \pmb u^{(m+1)}
\end{equation}
and on the boundary between element $(m-1)$ and $(m)$ is
\begin{equation}
  {\color{s} \mathsf{F}_\text{left}} = \pmb A^+ \pmb u^{(m-1)} + \pmb A^- \pmb u^{(m)}
\end{equation}

If the system is not linear,  from (\ref{eq:Adef})

\begin{equation}
  \int_{\pmb u^{(m)}}^{\pmb u^{(m+1)}} \pmb A d\pmb u= \pmb F (\pmb u^{(m+1)}) - \pmb F (\pmb u^{(m)})
\end{equation}
And the same expression for $\pmb F^+$ and  $\pmb F^-$.

The Solomon-Osher type flux is 
\begin{equation}
  {\color{s} \mathsf{F}_\text{right}} = \pmb F^+( \pmb u^{(m)} )+ \pmb F^- (\pmb u^{(m+1)}).
\end{equation}
or, if $\pmb F^+$ or  $\pmb F^-$ are expanded, it can be expressed in different forms:

\begin{align}\label{eq:solomonflux}
{\color{s} \mathsf{F}_\text{right}} = 
  \pmb F( \pmb u^{(m)} ) +  \int_{\pmb u^{(m)}}^{\pmb u^{(m+1)}} \pmb A^- (\pmb u) d\pmb u;\\
{\color{s} \mathsf{F}_\text{right}} = 
  \pmb F( \pmb u^{(m+1)} ) +  \int_{\pmb u^{(m)}}^{\pmb u^{(m+1)}} \pmb A^+(\pmb u)  d\pmb u;\\
{\color{s} \mathsf{F}_\text{right}} = 
  \frac 12  \left( 
    \pmb F( \pmb u^{(m)} ) + 
    \pmb F( \pmb u^{(m+1)} )
  \right) -  
  \int_{\pmb u^{(m)}}^{\pmb u^{(m+1)}} (\pmb A^+(\pmb u)  - \pmb A^-(\pmb u) ) d\pmb u.
\end{align}

We can directly compute $ \pmb F( \pmb u^{(m)}$,  $ \pmb F( \pmb u^{(m+1)}$. For the second term, we need to compute the integral. 

In the $Q$-dimensional phase space, there are two states $\pmb u^{(m)} $ and $\pmb u^{(m+1)} $. 
They should be connected by a thermodynamically correct path, and the integral has to follow the path. 
As a simple approximation, we take the path 
\begin{equation}
  \pmb \Psi(s) =  \pmb u^{(m)}+s( \pmb u^{(m+1)} -  \pmb u^{(m)}), \quad 0\le s\le 1.
\end{equation}
\begin{equation}
  \pmb \Psi(0) = \pmb u^{(m)}, \quad  \pmb \Psi(1) = \pmb u^{(m+1)} .
\end{equation}

The integral variable changes to 
\begin{equation}
  d\pmb u \to \D {\pmb \Psi} s ds = (  \pmb u^{(m+1)} -  \pmb u^{(m)} ) ds.
\end{equation}

The integral is taken with the Gauss-Legendre quadrature:
\begin{equation}
  \int_0^1  (\pmb A^+(\pmb \Psi (s))  - \pmb A^-(\pmb \Psi (s)) ds = 
  \sum_j w_j \left( 
     (\pmb A^+(\pmb \Psi (s_j))  - \pmb A^-(\pmb \Psi (s_j))
  \right)
\end{equation}

where $w_j$ and $s_j$ are the Gaussian weights and quadrature points respectively. 

\clearpage
\section{Full algorithm}\label{sec:algorithm}

The data type for the solution vector  $\color o u_p$, $p=1..N_Q$ is 

\begin{lstlisting}
  using SolutionVector = std::array<ftype,NQ>
\end{lstlisting}
Where \lstinline{ftype} is single or double precision floating point value.  

$\color{f} \mathsf w_{pl}$ is the DG decomposition of the solution vector $u_p$ in space. 
The data type for $w_{pl}$ is 
\begin{lstlisting}
  using xDGdecomposition = std::array<T,NBx>; // U[ibx][iq] 
\end{lstlisting}
where \lstinline{T} is the class that includes $\color o u_p$ and other coefficients and methods required by the physical model. 


First, $\color{f} \mathsf w_{pl}$ are known. 
\begin{enumerate}
  \item We obtain $q_{pl}$ with the ADER update. For this purpose, 
     \begin{itemize}
       \item The data type for $q_{pl}$ is 
         \begin{lstlisting}
    using xtDGdecomposition = std::array<T,NBt*NBx>; // q[ibt*NBx+ibx][iq] 
         \end{lstlisting}
      \item We use (\ref{eq:ADERiter}), and insert the dimension units $\Delta t$, $\Delta x$ :
\begin{equation}
 {\color{k} q_m ^{(i+1)} } =
 (K_{1}^{-1})_{mk} \left(
  {\color{f} W_k } - \frac {\Delta t} {\Delta x} K_{\xi, kl} 
 {\color{o} \mathcal{F}_{pl}  ( {\color{k} \pmb q_{l}^{(i)} })} +
  \Delta t
 {\color{o} \mathcal{S}_{pk}  ( {\color{k} \pmb q_{k}^{(i)} })}
{w}_{k_\xi} 
{w}_{k_\tau} 
 \right)
\end{equation}
      \item
        The matrices $ K^{-1}_{1,mk}$ (\lstinline{K1inv}), and $K_{2,ml} = K^{-1}_{1,mk} K_{\xi, kl}$  (\lstinline{K2}) are found once in the class constructor. The Armadillo library is used for matrix inversion. 
      \item the initial guess for the iterative procedure $q_{pl}^{(0)}$ is estimated as 
        \begin{equation}
          q_{pl_\xi l_\tau} = {\color{f} \mathsf w_{pl_\xi}} \quad \text{ for all $l_\tau$ }.
        \end{equation}
      \item The ${\color{f} W_k } = {\color{f} \mathsf w_{pk_\xi}}{\color{z}\psi_{k_\tau}(0)w_{k_\xi}}$ vector  is found once before the iterations. 
      \item In each iteration $i$,
        \begin{enumerate}
          \item Compute $\pmb F(\pmb q_l^{(i)})$, $\pmb S(\pmb q_l^{(i)})$.
          \item Evaluate $\pmb q_l^{(i+1)}$.
        \end{enumerate}
     \end{itemize} 
  \item We compute and integrate over time the flux between the pair of elements.
     \begin{itemize}
       \item The predictor solution $\pmb q_{l_\xi l_\tau}$ which was found for the left and the right elements are evaluated at the element boundary to get \lstinline{qL} and  \lstinline{qR} correspondingly. 
         \begin{align}
           q_{L,l_\tau} = q^{(m)(i)  }_{l_\xi l_\tau} \psi_{l_\xi} (\xi = 1);\\
           q_{R,l_\tau} = q^{(m+1)(i)}_{l_\xi l_\tau} \psi_{l_\xi} (\xi = 0);\\
          \end{align}
       \item
         The integrals in time are found with the Gauss quadrature
         \begin{equation}
           \int_0^1 d\tau {\color{s} \mathsf{F}_\text{right}} = \sum_{l_\tau} w_{l_\tau}  \mathsf{F}_\text{right}( q_{L,l_\tau}, q_{R,l_\tau}). 
         \end{equation}
       \item
         ${\color{s} \mathsf{F}_\text{right}( q_{L,l_\tau}, q_{R,l_\tau})}$ is found for each $l_\tau$ with the Solomon-Oscher expression (\ref{eq:solomonflux})
         \begin{equation}
{\color{s} \mathsf{F}_\text{right}} = 
  \frac 12  \left( 
    \pmb F( \pmb q_{L,l_\tau} ) + 
    \pmb F( \pmb q_{R,l_\tau} )
  \right) -  
           \int_{\pmb q_{L,l_\tau}}^{\pmb q_{R,l_\tau}} (\pmb A^+(\pmb u)  - \pmb A^-(\pmb u) ) d\pmb u.
         \end{equation}
         To do this, 
         \begin{enumerate}
           \item $\pmb A^+(\pmb u)  - \pmb A^-(\pmb u)$ ( \lstinline{Amod} ) is found for several ( \lstinline{Oorder+1}) Gaussian quadrature points $s_j$.
           \item The analytical expression for $\pmb A(u)$ is written in the class for the physical model. The Armadillo package is used to find the eigenvalues and eigenvectors of $\pmb A(\pmb \Psi (s_j))$, where $\Psi (s_j) =  \pmb q_{L,l_\tau} + s (   \pmb q_{R,l_\tau} -  \pmb q_{L,l_\tau} ), $ and to perform further algebra to find \lstinline{Amod}.
           \item The integral is computed with the Gaussian quadrature with the order \lstinline{Oorder}. 
             \begin{equation}
               \int_{\pmb q_{L,l_\tau}}^{\pmb q_{R,l_\tau}} (\pmb A^+(\pmb u)  - \pmb A^-(\pmb u) ) d\pmb u = \sum_j w_j  (\pmb A^+(\pmb  \Psi (s_j))  - \pmb A^-(\pmb  \Psi (s_j)) ) (   \pmb q_{R,l_\tau} -  \pmb q_{L,l_\tau} )
             \end{equation}
         \end{enumerate}
     \end{itemize} 
   \item We find the new $u^+_{pk}$ value.  We use (\ref{eq:update}), and insert the dimension coefficients $\Delta t$, $\Delta x$.
\begin{equation}
  {\color{k} \mathsf{u}^+_{pk_\xi} }
  = 
  {\color{f} \mathsf{w}_{pk_\xi} } 
  - 
  \frac{\Delta t}{\Delta x w_{k_\xi}}
  \left (
  {\color{z} \int_0^1 d\tau \left( 
  \left.
  {\color{s} \mathsf{F}_\text{right}  }
  \psi_{k_\xi} 
  \right|_{\xi=1}
  - 
  \left.
  {\color{s} \mathsf{F}_\text{left}  }
  \psi_{k_\xi} 
  \right|_{\xi=0}
  -
  \int_0^1 
  {\color{o} F_p (\pmb u)}
  \D{\psi_{k_\xi}}{\xi}
  d\xi
  \right) + } \Delta x
 {\color{z}  
  \int_0^1 d\xi
  \psi_{k_\xi} (\xi) 
  \int_0^1 d\tau}
  {\color{o} S_p}.
  \right)
\end{equation}
     \begin{itemize}
       \item ${\color{s} \mathsf{F}_\text{left}( q_{L,l_\tau}, q_{R,l_\tau})}$ is the ${\color{s} \mathsf{F}_\text{right}( q_{L,l_\tau}, q_{R,l_\tau})}$ of the neighboring element to the left. 
       \item The volume flux and the source terms are expanded as 
         \begin{align}
           {\color{o} F_{p} (\pmb u) = F_p(\pmb q_l) \psi_{l_\xi}(\xi) \psi_{l_\tau}(\tau) }\\
           {\color{o} S_{p} (\pmb u,\xi,\tau) =  S_p (\pmb q_l, 
                                            \xi_{l_\xi}, 
                                            \tau_{l_\tau}) 
                                      \psi_{l_\xi}(\xi) 
                                      \psi_{l_\tau}(\tau)}
         \end{align}
         where $l_\xi$, $l_\tau$ are the Gaussian quadrature points. 
       \item After integration in $\tau$, $\xi$ these become:
         ${\color{o}F_p (\pmb q_l)}{\color{z} I_{l_\xi k_\xi}}  w_{l_\tau}$, and
         ${\color{o}S_p (\pmb q_l,\xi_{l_\xi}, \tau_{l_\tau})}{\color{z} w_{k_\xi} w_{l_\tau}}$
               correspondingly.  The matrix 
               \begin{equation}
                 {\color{z} I_{l_\xi k_\xi} = 
                 \int_0^1 }
                 {\color{o} \psi_{l_\xi}(\xi) }
                 {\color{z}
                 \D{\psi_{k_\xi}}{\xi} d\xi = w_{l_\xi}\left. \D {\psi_{k_\xi}}{\xi}\right|_{\xi=\xi_{k_\xi}}.
                 }
               \end{equation}
               is denoted as \lstinline{I4volflux} and it is evaluated once in the class constructor method. 
       \item We remind that there are $N_{Bx}$ vector components, where $1\le k_\xi \le N_{Bx}$, and the sum over $l_\tau$ and $l_\xi$ is performed.
     \end{itemize} 
\end{enumerate}


\section{Code Sructure}\label{sec:algorithm}
The code includes 
\begin{itemize}
  \item the \lstinline{main.cpp} file with  the \lstinline{DumbserMethod} class, where the ADER predictor and the mesh update functions are implemented as methods;
  \item several physical models definitions: advection, seismic, etc.;
  \item the \lstinline{fluxes.cpp} file with the functions for evaluation of fluxes;
  \item the \lstinline{PMpolynoms.py} file which generates the coefficients that are obtained from the choice of the Lagrange polynomial basis;
  \item the \lstinline{plot_error.py} and \lstinline{plot_output.py} for visualization of results. 
  \item the \lstinline{Makefile}
\end{itemize}

The  \lstinline{DumbserMethod} class is a template class with parameters:
\begin{itemize}
  \item \lstinline{int Mx}, the order of basis polynomials in space; \lstinline{NBx = Mx+1}. 
  \item \lstinline{int Mt}, the order of basis polynomials in space; \lstinline{NBt = Mt+1}.
  \item \lstinline{typename T}, the order of basis polynomials in space;
  \item \lstinline{int N}, number of spatial elements.
\end{itemize}

\clearpage
The computation consists of several steps:

\begin{itemize}
  \item Initialize method. 
    \begin{lstlisting}
      DumbserMethod<1, 1, gas::Gas, 100> mesh_calc;
    \end{lstlisting}
  \item set the domain physical size and the time step through the courant number (speed assumed to be 1).
    \begin{lstlisting}
      mesh_calc.set_Lx_Courant(1,.2);
    \end{lstlisting}
  \item Initialization
    \begin{lstlisting}
      mesh_calc.init();
    \end{lstlisting}
  \item Update
    \begin{lstlisting}
      mesh_calc.update(istep);
    \end{lstlisting}
  \item ADER update 
    \begin{lstlisting}
      mesh_calc.ADER_update(istep);
    \end{lstlisting}
  \item dump results
    \begin{lstlisting}
      mesh_calc.print_all(istep);
    \end{lstlisting}
  \item  dump approximation error
    \begin{lstlisting}
      mesh_calc.print_error(istep);
    \end{lstlisting}
\end{itemize}

%      //eqs4testing::model.set(mesh_calc.Lx);// >>>>>>>>>>>>>>> ? <<<<<<<<<<<<<<<<
%
%      mesh_calc.init();
%      int istep = 0;
%      mesh_calc.print_all(istep);
%
%      for (; istep < floor(mesh_calc.Tmax/mesh_calc.dt+0.5); istep++) {
%        fmt::print("istep {}\n",istep);
%        mesh_calc.update(istep);
%        mesh_calc.print_all(istep);
%      }
%      mesh_calc.print_all(istep);
%      //mesh_calc.ADER_update(istep);
%      //istep++;
%      //mesh_calc.init(istep*mesh_calc.dt);
%      //mesh_calc.print_all(-istep);
%      mesh_calc.print_error(istep);

\clearpage
\section{Physical models}\label{sec:models}

\subsection{Equations for testing}\label{sec:eqs4testing}

\begin{align}
  \D u t + \D {}x \left(
    \frac 12 v^2
  \right) = - \nu (u - u_e) + 
   \D {u_e} t + \D {}x \left(
    \frac 12 {v_e}^2
  \right)\\
  \D v t + \D {}x \left(
    \frac 12 u^2
  \right) = - \nu (v - v_e) + 
   \D {v_e} t + \D {}x \left(
    \frac 12 {u_e}^2
  \right)
\end{align}

\begin{align}
  u_e (x,t) = U_0 + A_u\sin(kx-\omega t), \\
  v_e (x,t) = V_0 + A_v\cos(kx-\omega t).
\end{align}

\begin{equation}
  \nu =10; \quad
  U_0 =4; \quad
  V_0 =6; \quad
  A_u =0.1; \quad
  A_v =0.3
\end{equation}

\begin{equation}
  {\color{o} \pmb u} = 
  \begin{pmatrix}
    u\\ v
  \end{pmatrix}; \quad
  {\color{o} \pmb F (\pmb u )} = 
  \begin{pmatrix}
    \frac 12 v^2\\ 
    \frac 12 u^2
  \end{pmatrix}; \quad
  {\color{o} \pmb S (\pmb u )} = 
  \begin{pmatrix}
    -\nu(u-u_e) +
    \D {u_e} t + \D {}x \left(
    \frac 12 {v_e}^2
    \right)
    \\ 
    -\nu(v-v_e) +
    \D {v_e} t + \D {}x \left(
    \frac 12 {u_e}^2
    \right)
  \end{pmatrix}; \quad
\end{equation}

\begin{equation}
  \pmb A =
  \begin{pmatrix}
    0 & v\\
    u & 0
  \end{pmatrix}.
\end{equation}

\begin{equation}
  \pmb A = \pmb K \pmb \Lambda \pmb K^{-1}.
\end{equation}

\begin{equation}
  \pmb \Lambda =
  \begin{pmatrix}
    -\sqrt{uv} & 0\\
    0 & \sqrt{uv}
  \end{pmatrix}; \quad
  \pmb K =
  \begin{pmatrix}
    1 & 1\\
    -\sqrt{\frac vu} & 
    \sqrt{\frac vu} & 
  \end{pmatrix}; 
\end{equation}

\subsection{seismic}\label{sec:seismic}

\begin{align}
  &\D {\sigma_x} t = (\lambda+2\mu) \D {v_x} x + S_1(x,t);\\
  &\rho \D {v_x} t =  \D {\sigma_x} x + S_2(x,t)
\end{align}

\begin{equation}
  c_p = \sqrt{
    \frac {\lambda +2\mu}\rho
  }
\end{equation}

\begin{equation}
  {\color{o} \pmb u} = 
  \begin{pmatrix}
    \sigma_x \\ v_x
  \end{pmatrix}; \quad
  {\color{o} \pmb F (\pmb u )} = 
  \begin{pmatrix}
   - \rho c_p^2 \\ 
   - \frac 1 \rho
  \end{pmatrix}.
\end{equation}

\begin{equation}
  \pmb A =
  \begin{pmatrix}
    0 & -\rho c_p^2\\
    -\frac 1\rho & 0
  \end{pmatrix}.
\end{equation}

\begin{equation}
  \pmb A = \pmb K \pmb \Lambda \pmb K^{-1}.
\end{equation}

\begin{equation}
  \pmb \Lambda =
  \begin{pmatrix}
    -c_p & 0\\
    0 & c_p
  \end{pmatrix}; \quad
  \pmb K =
  \begin{pmatrix}
    1 & 1\\
    -\frac {c_p}\rho & 
     \frac {c_p}\rho & 
  \end{pmatrix}; 
\end{equation}

\begin{equation}
  \pmb \Lambda^- =
  \begin{pmatrix}
    -c_p & 0\\
    0 & 0
  \end{pmatrix}; \quad
  \pmb \Lambda^+ =
  \begin{pmatrix}
    0 & 0\\
    0 & c_p
  \end{pmatrix}
\end{equation}

\begin{equation}
  \pmb F^+ (\pmb u ) =  \pmb A^+ \pmb u 
  = \pmb K \pmb \Lambda^+ \pmb K^{-1} = \frac 12
  \begin{pmatrix}
     -\rho c_p^2 v_x + c_p \sigma_x \\
     c_p v_x - \frac 1\rho \sigma_x \\
  \end{pmatrix}.
\end{equation}

\begin{equation}
  \pmb F^- (\pmb u ) =  \pmb A^- \pmb u 
  = \pmb K \pmb \Lambda^- \pmb K^{-1} = \frac 12
  \begin{pmatrix}
     -\rho c_p^2 v_x - c_p \sigma_x \\
     - c_p v_x - \frac 1\rho \sigma_x \\
  \end{pmatrix}.
\end{equation}


\begin{equation}
  \pmb S (x,t) = 
  = \pmb K_i s(t) \delta(x-x_0),\quad i=1,2.
\end{equation}
\subsection{seismic}\label{sec:seismic}


\section{Future Work}
\begin{itemize}
  \item A posteriori limiter
  \item AMR?
  \item Non-conservative Baer-Nunziato model
  \item 3D
\end{itemize}


\clearpage
\section{DG with non-conservative terms} \label{sec:DGnoncons}

Let us take a system of equations  for vector of variables $\color{o}{\mathbf{u}}$, its components are $u_p$, $p=1..N_Q$. 
\begin{equation} \label{init}
 \D {\color{o} u_p} {t} + 
 \D {\color{o} F_p (\pmb u)}{x} +
  {\color{o} B_{pq}(\pmb u)} \D {\color{o} u_q} {x}
  = 
  {\color{o} S_p(\pmb u, x, t)} \quad
\end{equation}

We introduce dimensionless fluxes and sources as:
\begin{equation} 
  \pmb F^* = \frac{\Delta t}{\Delta x} \pmb F; \quad
  \pmb B^* = \frac{\Delta t}{\Delta x} \pmb B; \quad
  \pmb S^* =       \Delta t            \pmb S.
\end{equation}

The system with   $\pmb F^*$ and $\pmb S^*$ is as follows. We multiply it by a basis function (Lagrange polynomial) and integrate over the element. 
\begin{equation} \label{DG}
\left.
\left.
 \D {\color{o} u_p} {\tau} + 
 \D {\color{o} F^*_p (\pmb u)}{\xi} 
 +
  {\color{o} B^*_{pq}(\pmb u)} \D {\color{o} u_q} {\xi}
  = 
  {\color{o} S^*_p(\pmb u, \xi(x), \tau(t))} \quad
 \right| 
 {\color{z} \cdot
  \psi_{k_\xi} (\xi) }
 \right| 
 {\color{z} \cdot 
  \int_0^1 d\xi
  \int_0^1 d\tau}
\end{equation}

The first term is integrated in $\tau$. Expressed trhough the values stored on the mesh before and after the DG update:

\begin{equation} 
  \left. u_p \right|_{\tau=0}\approx  {\color{f} \mathsf{w}_{pl_\xi} }
   {\color{o} \psi_{l_\xi} (\xi)}
\end{equation}
\begin{equation} 
  \left. u_p \right|_{\tau=1}\approx  {\color{k} \mathsf{u}^+_{pl_\xi} }
   {\color{o} \psi_{l_\xi} (\xi)}
\end{equation}

In the other terms, the integral in time is evaluated later. We need to insert the dependence on time. 
The dependence of $u_p$ in time is approximated with ADER (next chapter) to be

\begin{equation} 
  u_p \approx q_{pl_\xi l_\tau} 
   {\color{o} \psi_{l_\xi} (\xi)}
   {\color{o} \psi_{l_\tau} (\tau)}
\end{equation}

\begin{equation} 
  {\color{o} B^*_{pq}(\pmb u)}  \approx {\color{o} B_{pql_\xi l_\tau}}
   {\color{o} \psi_{l_\xi} (\xi)}
   {\color{o} \psi_{l_\tau} (\tau)}
   = 
  {\color{o} B^*_{pq}(\pmb u)}  = {\color{o} B_{pq}(\pmb q_{l_\xi l_\tau})
   {\color{o} \psi_{l_\xi} (\xi)}
   {\color{o} \psi_{l_\tau} (\tau)}
  }
\end{equation}


Let us inspect the smooth part of the non-conservative product. If $B_{pq}=\color{o} B_{pql_\xi l_\tau} \psi_{l_\xi} \psi_{l_\tau}$, $q_q=\color{k} q_{qm_\xi m_\tau} \psi_{m_\xi} \psi_{m_\tau}$,
\begin{equation}
  \int_0^1 
  \int_0^1
  {\color{o} B_{pql_\xi l_\tau} \psi_{l_\xi} \psi_{l_\tau}}
  {\color{k} q_{qm_\xi m_\tau} \D{\psi_{m_\xi}}{\xi} \psi_{m_\tau}}
  {\color{z} \psi_{k_\xi} }
  d\tau 
  d\xi  = ?
\end{equation}

\begin{equation}
  \int_0^1
  {\color{o} \psi_{l_\tau}}
  {\color{k} \psi_{m_\tau}}
  d\tau  = w_{l_\tau} \delta_{l_\tau m_\tau}
\end{equation}

\begin{equation}
  \int_0^1
  {\color{o} \psi_{l_\xi} }
  {\color{k} \D{\psi_{m_\xi}}{\xi} }
  {\color{z} \psi_{k_\xi} }
  d\xi
  = 
  \sum_i w_i
  {\color{o} \delta_{l_\xi i} }
  {\color{k}\left. \D{\psi_{m_\xi}}{\xi} \right|_{\xi=\xi_i}}
  {\color{z} \delta_{k_\xi i} }
  = 
  w_{k_\xi}
  {\delta_{k_\xi l_\xi} }
  {\left. \D{\psi_{m_\xi}}{\xi} \right|_{\xi=\xi_{k_\xi}}}
\end{equation}

\begin{equation}
  \int_0^1 
  \int_0^1
  {\color{o} B_{pql_\xi l_\tau} \psi_{l_\xi} \psi_{l_\tau}}
  {\color{k} q_{qm_\xi m_\tau} \D{\psi_{m_\xi}}{\xi} \psi_{m_\tau}}
  {\color{z} \psi_{k_\xi} }
  d\tau 
  d\xi
  =
  {\color{o} B_{pqk_\xi l_\tau}}
  {\color{k} q_{qm_\xi l_\tau} }
  w_{l_\tau} 
  w_{k_\xi}
  {\left. \D{\psi_{m_\xi}}{\xi} \right|_{\xi=\xi_{k_\xi}}}
\end{equation}

The intercell flux and the jump part of non-conservative product is merged together into $\mathcal{D}(q^-,q^+)$, to be defined later. 

The expression for volumetric flux is
         $F_p (\pmb q_l) I_{l_\xi k_\xi}  w_{l_\tau}$, and
               the matrix 
               \begin{equation}
                 {\color{z} I_{l_\xi k_\xi} = 
                 \int_0^1 }
                 {\color{o} \psi_{l_\xi}(\xi) }
                 {\color{z}
                 \D{\psi_{k_\xi}}{\xi} d\xi = w_{l_\xi}\left. \D {\psi_{k_\xi}}{\xi}\right|_{\xi=\xi_{k_\xi}}.
                 }
               \end{equation}

The explicit expression for the update of $u_p$ is:

\begin{align}
  {\color{k} \mathsf{u}^+_{pk_\xi} }
  w_{k_\xi}  = 
  {\color{f} \mathsf{w}_{pk_\xi} } 
  w_{k_\xi}  
  + F_{pl_\xi l_\tau} I_{l_\xi k_\xi}  w_{l_\tau}
  - {\color{o} B_{pqk_\xi l_\tau}}
  {\color{k} q_{qm_\xi l_\tau} }
  w_{l_\tau} 
  w_{k_\xi}
  {\left. \D{\psi_{m_\xi}}{\xi} \right|_{\xi=\xi_{k_\xi}}} -
  \\
  - \left( 
   \left. 
     {\color{s} \int_0^1 d\tau  \mathcal{D} (\pmb q^-_{l_\tau},\pmb q^{\text{right}+}_{l_\tau}) } \psi_{k_\xi}
   \right|_{\xi=1} 
    - 
   \left. 
     {\color{s} \int_0^1 d\tau \mathcal{D} (\pmb q^{\text{left}-}_{l_\tau},\pmb q^+_{l_\tau}) } \psi_{k_\xi} 
   \right|_{\xi=0} 
  \right)
\end{align}

\end{document}

